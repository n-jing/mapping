\documentclass{article}

\usepackage{subfigure}
\usepackage{graphicx}             %添加图片
\graphicspath{{./figs/}{./draw/}{./screen/}}
\usepackage{bm}                   %专门处理数学粗体的bm宏包,使用命令是\bm{}



\input{structure_report.tex} % Include the file specifying the document structure and custom commands

%----------------------------------------------------------------------------------------
%	ASSIGNMENT INFORMATION
%----------------------------------------------------------------------------------------

\usepackage{setspace}
\renewcommand{\baselinestretch}{1.2}

\title{Parameterization} % Title of the assignment

\author{Jing JIANG\\ \texttt{siliuhe@sina.com}} % Author name and email address

\date{%% State Key Lab of CAD\&CG,Zhejiang University --- 
  \today} % University, school and/or department name(s) and a date

%----------------------------------------------------------------------------------------

\begin{document}

\maketitle % Print the title

%----------------------------------------------------------------------------------------
%	INTRODUCTION
%----------------------------------------------------------------------------------------

\section*{Barycentric Mapping} % Unnumbered section
\subsection*{Introduction}
According to Tutte��s barycentric mapping theorem,
\begin{quote}
  Given a triangulated surface homeomorphic to a disk, if the ($u, v$) coordinates at the boundary vertices lie on a convex polygon, and if the coordinates of the internal vertices are a convex combination of their neighbors, then the ($u, v$) coordinates form a valid parameterization (without self-intersections).
\end{quote}

$$\forall i \in {1,2,...,n_{in}}, \qquad -a_{i, i}
\left( \begin{array}{c}
         u_i \\
         v_i
       \end{array}\right)
     =\sum_{j\neq i}a_{i, j}
     \left( \begin{array}{c}
       u_i \\
       v_i
            \end{array}\right) $$
\subsection*{Result}
result

\end{document}
