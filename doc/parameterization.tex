\documentclass{article}

\usepackage{subfigure}
\usepackage{graphicx}             %添加图片
\graphicspath{{./figs/}{./draw/}{./screen/}}
\usepackage{bm}                   %专门处理数学粗体的bm宏包,使用命令是\bm{}



\input{structure_report.tex} % Include the file specifying the document structure and custom commands

%----------------------------------------------------------------------------------------
%	ASSIGNMENT INFORMATION
%----------------------------------------------------------------------------------------

\usepackage{setspace}
% \renewcommand{\baselinestretch}{1.2}

\title{Parameterization} % Title of the assignment

\author{Jing JIANG\\ \texttt{siliuhe@sina.com}} % Author name and email address

\date{%% State Key Lab of CAD\&CG,Zhejiang University --- 
  \today} % University, school and/or department name(s) and a date

%----------------------------------------------------------------------------------------

\begin{document}

\maketitle % Print the title

%----------------------------------------------------------------------------------------
%	INTRODUCTION
%----------------------------------------------------------------------------------------

\section*{Barycentric Mapping} % Unnumbered section
\subsection*{Introduction}
According to Tutte��s barycentric mapping theorem,
\begin{quote}
  Given a triangulated surface homeomorphic to a disk, if the ($u, v$) coordinates at the boundary vertices lie on a convex polygon, and if the coordinates of the internal vertices are a convex combination of their neighbors, then the ($u, v$) coordinates form a valid parameterization (without self-intersections).
\end{quote}

$$\forall i \in {1,2,...,n_{in}}, \qquad -a_{i, i}
\left( \begin{array}{c}
         u_i \\
         v_i
       \end{array}\right)
     =\sum_{j\neq i}a_{i, j}
     \left( \begin{array}{c}
       u_i \\
       v_i
\end{array}\right) $$

\subsection*{Implement}
\begin{enumerate}
  \item judge whether the model is a patch
  \item build half-edge structure
  \item get the boundary edges and the texture coordinates of the boundary vertices
  \item get the texture coordinates of the inner vertices
\end{enumerate}
\subsection*{Result}
\begin{figure}[htbp]
  \centering
  \subfigure[]{
    \begin{minipage}[t]{0.24\linewidth}
      \centering
      \includegraphics[width=3.2cm]{patch_1_o}
      % \caption{fig1}
    \end{minipage}%
  }%
  \subfigure[]{
    \begin{minipage}[t]{0.24\linewidth}
      \centering
      \includegraphics[width=3.2cm]{patch_1_t}
      % \caption{fig2}
    \end{minipage}%
  }%
  \subfigure[]{
    \begin{minipage}[t]{0.24\linewidth}
      \centering
      \includegraphics[width=3.2cm]{patch_2_o}
      % \caption{fig1}
    \end{minipage}%
  }%
  \subfigure[]{
    \begin{minipage}[t]{0.24\linewidth}
      \centering
      \includegraphics[width=3.2cm]{patch_2_t}
      % \caption{fig2}
    \end{minipage}%
  }%
  \centering
  \caption{barycentering mapping, a and c are origin models, b and d are models with texture coordinates}
\end{figure}

\section*{Least Squares Conformal Mapping} % Unnumbered section
\subsection*{Introduction}

\begin{figure}[htbp]
  \centering
  \subfigure[]{
    \begin{minipage}[t]{0.48\linewidth}
      \centering
      \includegraphics[width=6.2cm]{cm}
      % \caption{fig1}
    \end{minipage}%
  }%
  \subfigure[]{
    \begin{minipage}[t]{0.48\linewidth}
      \centering
      \includegraphics[width=6.2cm]{xy}
      % \caption{fig2}
    \end{minipage}%
  }%
  \centering
  \caption{a: mapping $\bm{X}$ to $\bm{u}$. b: local X, Y basis in a triangle}
\end{figure}

\noindent Conformal mapping: the anisotropy ellipse is a circle for all points of the surface.

\noindent Conformal condition: $\bm{X}_{\mu}=\bm{n}\times\bm{X}_u$, $\nabla {\mu}=\bm{n}\times\nabla{u}$

$$
\nabla u=\left[\begin{array}{l}
{\partial u / \partial X} \\
{\partial u / \partial Y}
\end{array}\right]=\underbrace{\frac{1}{2 A_{T}}\left[\begin{array}{lll}
{Y_{j}-Y_{k}} & {Y_{k}-Y_{i}} & {Y_{i}-Y_{j}} \\
{X_{k}-X_{j}} & {X_{i}-X_{k}} & {X_{j}-X_{i}}
\end{array}\right]}_{=\mathbf{M}_{T}}\left(\begin{array}{l}
{u_{i}} \\
{u_{j}} \\
{u_{k}}
\end{array}\right)
$$

conformal condition, 
$$
\nabla v=(\nabla u)^{\perp}=\left[\begin{array}{cc}
{0} & {-1} \\
{1} & {0}
\end{array}\right] \nabla u
$$

$$
\mathbf{M}_{T}\left(\begin{array}{l}
{v_{i}} \\
{v_{j}} \\
{v_{k}}
\end{array}\right)-\left[\begin{array}{cc}
{0} & {-1} \\
{1} & {0}
\end{array}\right] \mathbf{M}_{T}\left(\begin{array}{l}
{u_{i}} \\
{u_{j}} \\
{u_{k}}
\end{array}\right)=\left(\begin{array}{l}
{0} \\
{0}
\end{array}\right)
$$

energe to be optimed, 
$$
E_{\mathrm{LSCM}}=\sum_{T=(i, j, k)} A_{T}\left\|\mathbf{M}_{T}\left(\begin{array}{c}
{v_{i}} \\
{v_{j}} \\
{v_{k}}
\end{array}\right)-\left[\begin{array}{cc}
{0} & {-1} \\
{1} & {0}
\end{array}\right] \mathbf{M}_{T}\left(\begin{array}{c}
{u_{i}} \\
{u_{j}} \\
{u_{k}}
\end{array}\right)\right\|^{2}
$$

\subsection*{Result}
\end{document}
